\documentclass{article}

\usepackage{amsmath, amsthm, amssymb, amsfonts}
\usepackage{thmtools}
\usepackage{graphicx}
\usepackage{setspace}
\usepackage{geometry}
\usepackage{float}
\usepackage{hyperref}
\usepackage[utf8]{inputenc}
\usepackage[english]{babel}
\usepackage{framed}
\usepackage[dvipsnames]{xcolor}
\usepackage{tcolorbox}

\colorlet{LightGray}{White!90!Periwinkle}
\colorlet{LightOrange}{Orange!15}
\colorlet{LightGreen}{Green!15}

\newcommand{\HRule}[1]{\rule{\linewidth}{#1}}

\declaretheoremstyle[name=Method,]{thmsty}
\declaretheorem[style=thmsty,numberwithin=section]{theorem}
\tcolorboxenvironment{theorem}{colback=LightGray}

\declaretheoremstyle[name=proposition,]{prosty}
\declaretheorem[style=prosty,numberlike=theorem]{proposition}
\tcolorboxenvironment{proposition}{colback=LightOrange}

\declaretheoremstyle[name=Principle,]{prcpsty}
\declaretheorem[style=prcpsty,numberlike=theorem]{principle}
\tcolorboxenvironment{principle}{colback=LightGreen}

% a style for problems that has its own counter within each subsection
\declaretheoremstyle[name=Problem,]{probsty}
\declaretheorem[style=probsty,numberwithin=subsection]{problem}
\tcolorboxenvironment{problem}{colback=LightOrange}

% a style for solutions that has the counter as problems and the background is white but with borders and the text starts in the next
\declaretheoremstyle[name=Sol,]{solsty}
\declaretheorem[style=solsty,numbered=no]{solution}
\tcolorboxenvironment{solution}{colback=white, colframe=Gray, boxrule=1pt, title=Solution }

\setstretch{1.2}
\geometry{
    textheight=9in,
    textwidth=5.5in,
    top=1in,
    headheight=12pt,
    headsep=25pt,
    footskip=30pt
}

\makeatletter
\newcommand*{\rom}[1]{\expandafter\@slowromancap\romannumeral #1@}
\makeatother




% ------------------------------------------------------------------------------

\begin{document}

% ------------------------------------------------------------------------------
% Cover Page and ToC
% ------------------------------------------------------------------------------

\title{ \normalsize \textsc{}
		\\ [2.0cm]
		\HRule{1.5pt} \\
		\LARGE \textbf{\uppercase{Module 1: Numerical Methods}
		\HRule{2.0pt} \\ [0.6cm] \LARGE{Numerical solution of ordinary differential equations of first order and first degree,
        Numerical solution of algebraic and transcendental equations} \vspace*{10\baselineskip}}
		}
\date{}
\author{\textbf{Editor} \\ 
		Mohammed Ali Al sakkaf \\
		BMSIT \\
		22/09/2021}

\maketitle
\newpage

\tableofcontents
\newpage

% ------------------------------------------------------------------------------

\section{Numerical solution of ordinary differential equations of first order and first degree}

\subsection{Taylor’s series method}

\begin{theorem}
    Consider a differential equation
    \begin{equation}
        \frac{dy}{dx} = f(x, y) , \quad y(x_0) = y_0
    \end{equation}
    The Taylor's Series Solution is:
    \begin{equation}
        y = y(x_0) + (x - x_0) y^{'}(x_0) + \frac{(x - x_0)^2}{2!} y^{''}(x_0) + \frac{(x - x_0)^3}{3!} y^{'''}(x_0) + \cdots
    \end{equation}
\end{theorem}

% problems for this method -------
\begin{problem}
    Solve the following differential equation using Taylor's Series Method at $x = 0.1$:
    \begin{equation*}
        \frac{dy}{dx} = x - y^2 , \quad y(0) = 1
    \end{equation*}
\end{problem}

% header for the solution and the solution itself and make it left aligned
\begin{solution}
    Given that: $y^{'} = x - y^2 , \quad x_0 = 0 , \quad y_0 = 1$\\
    $\therefore y^{'} = x - y^2 , \quad y^{'}(0) = -1$\\
    $\Rightarrow y^{''} = 1 - 2y y^{'} , \quad y^{''}(0) = 1$\\
    $\Rightarrow y^{'''} = -2y^{'} y^{'} - 2y y^{''} , \quad y^{'''}(0) = -2$\\
    $\Rightarrow y^{\rom{4}} = -2y^{'} y^{''} - 2y^{'} y^{''} - 2y y^{'''} - 2y^{'} y^{''} , \quad y^{\rom{4}}(0) = -4$\\
    By Taylor's Series Method:\\
    $y = y_0 + (x - x_0) y^{'}(x_0) + \frac{(x - x_0)^2}{2!} y^{''}(x_0) + \frac{(x - x_0)^3}{3!} y^{'''}(x_0) + \cdots$\\
    $\Rightarrow y = 1 + (x - 0) (-1) + \frac{(x - 0)^2}{2!} (1) + \frac{(x - 0)^3}{3!} (-2) + \frac{(x - 0)^4}{4!} (-4)$\\
    $\Rightarrow y = 1 - x + \frac{x^2}{2} - \frac{x^3}{3} + \frac{x^4}{4}$\\
    at $x = 0.1$\\
    $\Rightarrow y = 1 - 0.1 + \frac{0.1^2}{2} - \frac{0.1^3}{3} + \frac{0.1^4}{4}$\\
    $\Rightarrow y = 0.904$
\end{solution}

\subsection{modified Euler’s method}


% ------------------------------------------------------------------------------

\section{Examples}

\begin{theorem}
    This is a theorem.
\end{theorem}

\begin{proposition}
    This is a proposition.
\end{proposition}

\begin{principle}
    This is a principle.
\end{principle}

% Maybe I need to add one more part: Examples.
% Set style and colour later.

\subsection{Pictures}

\begin{figure}[htbp]
    \center
    \includegraphics[scale=0.06]{img/photo.png}
    \caption{Sydney, NSW}
\end{figure}

\subsection{Citation}

This is a citation\cite{Eg}.

\newpage

% ------------------------------------------------------------------------------
% Reference and Cited Works
% ------------------------------------------------------------------------------

\bibliographystyle{IEEEtran}
\bibliography{References.bib}

% ------------------------------------------------------------------------------

\end{document}